\documentclass[12pt]{article}
\usepackage{setspace}
\usepackage{parskip}
\usepackage[margin=1in]{geometry}
\doublespacing
\begin{document}
\begin{flushleft}
\noindent
Jorie Sieck \\
HST 390-A \\
10/1/2020 \\
\textit{I pledge my honor that I have abided by the Stevens Honor System.}
\begin{center} \large Mental Health of College Students During COVID-19 \end{center}
\setlength{\parindent}{.5in}

The COVID-19 pandemic has exacerbated, and to some extent shed light on, many broader societal issues. One such issue is mental health and access to mental health treatment. For many students, a virtual or hybrid semester presents unique challenges that lead to worsening mental health and decreased performance. University mental health treatment is often subpar at best, and when students are not on campus, getting treatment becomes even more complicated. Even if a student is not struggling enough to need treatment, the harsh reality of this semester is that it is simply more difficult than normal. At a time when both students and professors are facing challenges they have never dealt with before, expecting a pre-COVID level of engagement and productivity from either group is unfair to everyone involved.

While mental health treatment, or lackthereof, is certainly not an issue unique to college students, there are several factors that can make finding appropriate treatment harder for students. For example, many college students are on their parents' health insurance plans. If their school is in a different state, then their options for treatment are limited to providers who accept out of area insurance, or whatever treatment they may have access to through their school. University mental health treatment is notoriously unreliable, however, and many university mental health centers cannot prescribe medication.

In some ways, the sudden popularity of virtual meetings makes mental health treatment more accessible for college students. Many mental health professionals now offer telehealth appointments, meaning that a student living out of state may still have access to in-network treatment back home. However, this poses other challenges: what if the student has no private space from which they can participate in a sensitive meeting such as therapy? What if they need medication that poses logistical challenges when prescribed across state lines? All of these factors can make getting appropriate mental health treatment incredibly difficult.

There are also many students this semester that do not require mental health treatment, but are still struggling more than usual. For example, every student I spoke to mentioned experiencing increased difficulty in being attentive during their lectures this semester (BU and WSU Interviews 2020). One student cited the fact that most of his lectures are asynchronous as a driving factor in his inattention, noting that the lack of a set schedule makes it hard to remember to watch the videos. He also mentioned that his professors seem to struggle without the visual feedback from students that they would receive during a live lecture (WSU Interview 2020).

Even during live lectures, however, there are many factors of virtual learning that make attentiveness difficult. For example, I find it easy to forget to silence my notifications during virtual lectures, which create a significant external distraction. During lectures that do not require my video to be on, I am also more likely to check notifications I receive because I do not have the social pressure of not wanting to appear distracted to the lecturer.

In addition to dealing with external distractions, some students report an increase in mental exhaustion this semester. One student mentioned considerable difficulty concentrating in lectures even when she is actively trying to pay attention (BU Interview 2020). This is something I deal with a lot, as well. I frequently find myself concentrating so much on paying attention to the lecture that I end up not actually concentrating on the lecture itself. While this issue is not unique to this semester for me, it has gotten notably worse with the transition to online lectures.

One possible cause for this mental exhaustion in lectures is ``Zoom fatigue," a term coined in recent months to describe the phenomenon in which many people find themselves significantly more tired after a video chat than a face to face one. Dr. Andrew Franklin of Norfolk State University explains that video calls force one's brain to work harder, resulting in increased exhaustion. Franklin poses multiple solutions to alleviate ``Zoom fatigue", such as conducting meetings via phone instead of video chat or turning off one's video during a call (Sklar 2020). While these suggestions may be helpful to people with jobs working remotely, they are less so for college students who do not have control over the way that their classes meet.

Outside of lectures, most students find this semester to be much more difficult than previous semesters. One student noted that she finds herself spending much more time studying now than she did pre-COVID, largely rewatching lectures to make up for the information she did not retain on the first watch (BU Interview 2020). My personal experience has been similar. I frequently take much longer to complete assignments than is typical, because I often have to spend a significant amount of time even just trying to parse the assignment specifications. This has resulted in a notable reduction in my productivity because I expend so much mental energy trying to figure out what I am supposed to be doing that I then need a break before I can begin to actually complete the assignment. The assignments themselves are also much more difficult than they would be in a normal semester.

Many of the issues identified above are largely unavoidable as long as the pandemic continues. The challenges of virtual instruction certainly outweigh the health risk that in person classes would have posed. And some issues are up to the students ourselves to solve. We are in charge of whether or not we check our notifications while in lecture, for example. But it is clear that this semester is harder than ever before for most students, and yet in many cases the same level of productivity is expected with no increase in support systems.

\pagebreak
\begin{center}\large References \end{center}
\setlength{\parindent}{0in}
\hangindent=4em{Interview with an undergraduate Engineering major at Wright State University, September 2020.} \\
\hangindent=4em{Interview with an undergraduate Biochemistry and Pre-Med major at Boston University, September 2020.} \\
\hangindent=4em{Sklar, Julia. ```Zoom Fatigue' Is Taxing the Brain. Here's Why That Happens." {\it National Geographic}. National Geographic Partners, LLC, April 24, 2020. https://www.nationalgeographic.com/science/2020/04/coronavirus-zoom-fatigue-is-taxing-the-brain-here-is-why-that-happens/.}

\end{flushleft}
\end{document}
